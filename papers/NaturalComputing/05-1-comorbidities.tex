% !TEX root =  ./main.tex

\subsection{Comorbidity Treatment Analysis}

The paper~\cite{DBLP:conf/cmsb/BowlesBBFGM24} introduced the notion of RS with guarded contexts to model scenarios where the entities to be provided by the context are dependent upon the current state, which is a common situation arising, e.g., in drug administration and \emph{in silico} experiments. The motivating case study considered in~\cite{DBLP:conf/cmsb/BowlesBBFGM24} is the risk reduction of medication harm in the treatment of patients with comorbidities. They are patients diagnosed with two or more long-term chronic conditions (such as diabetes, hypertension, cardiovascular diseases, chronic kidney disease, cancer, chronic obstructive pulmonary disease, among many others) and who are therefore subject to follow several treatment plans simultaneously, called \emph{clinical guidelines}~\cite{feder1999using,woolf1999potential}. Since clinical guidelines address a single disease, comorbidities easily lead to  \emph{polypharmacy}, where 5 or more medications must be administered, increasing the risk of adverse drug reactions, or of making certain drugs to be less effective when combined~\cite{Gut12}. Using formal methods for risk mitigation intends to help doctors choose between alternative treatment options as well as to point out missing conditions that could be helpful to revise and update clinical guidelines. 

\subparagraph*{Features of interest.}
In this case study, reachability and causal analysis are still key issues (e.g., can the combination of clinical guidelines expose the patient at serious risks because of drugs interference? If so, which medical decisions would be directly responsible? Would any alternative be available?
We have selected this case study because the use of guarded context poses new challenges in terms of causal analysis, as an external choice, apparently made by the context, might in fact be the only option available given the actual state of the system. 

\subparagraph*{Experimental set up.}
The RS encoding proposed in~\cite{DBLP:conf/cmsb/BowlesBBFGM24} relies on a formal representation of patient profiles, medical guidelines, adverse drug reactions.

For medical guidelines, it takes in input the event structure modelling of therapies introduced in~\cite{BC17c}. Roughly, to each event $e$ there is an identifier $\mathsf{E}_e$ defined as a sum of processes, one for each outgoing arc of $e$. If some guard is attached to the arc, then the corresponding alternative is also guarded. When some drug prescription for $d$ is present in a node, then the corresponding choice produces $\mathsf{get}\_d$. Similarly, if the therapy requires stopping some drug $d$, the entity $\mathsf{stop}\_d$ is produced.
Concurrent events are activated separately. 

The patient profile is determined by the conditions that trigger the treatment (e.g., headache, hypertension) and by the conditions that appear in the arc labels of the event structure (e.g., pregnant, asthma). We call them \emph{features}. The patient profile is thus just a combination of features. Correspondingly, there is one context $\mathsf{K}_f = \emptyset.\mathsf{Empty} + \{f\}.\mathsf{Empty}$ for each feature $f$, and any possible combination of features is accounted for by their parallel composition $\prod_f \mathsf{K}_f$. Alternatively, specific profiles can be investigated by removingsome alternatives from each $\mathsf{K}_f$.
Once the profile is determined by the context, it is preserved during the rest of the computation by reactions of the form $(\{f\},\emptyset,\{f\})$, one for each feature.

For each drug $d$ that appears in the therapies, we consider three corresponding entities $\mathsf{get}\_d$, $\mathsf{stop}\_d$  and $d$: the first represents the prescription of $d$ by the doctor, the second the removal of $d$ from the current treatment and the third the intake of the drug by the patient. 
Entities $\mathsf{get}\_d$ and $\mathsf{stop}\_d$ will be provided by the context that models the guideline. 
For each drug, $d$ there will be the following reactions: $(\{\mathsf{get}\_d\},\{\mathsf{stop}\_d,\mathsf{stop}\_c\},\{d,c\})$ modelling the intake of the drug $d$ as for doctor prescription, and $(\{d\},\{\mathsf{stop}\_d,\mathsf{stop}\_c\},\{d,c\})$ modelling the prosecution of the therapy. 
Adverse drug reactions are taken in tabular representation, called ADR tables.
Each row corresponds to a set of medications $M$, a textual description of their side effects and risks when used in combination, and a severity level $m$ (e.g., minor, moderate, major).
To each entry, there is a reaction $(M,\emptyset,\{m\})$. 

\subparagraph*{Analysis goals.}
The goal of the analysis is to explore the combination of clinical guidelines in the presence of comorbidities and for different patient profiles to detect if major risks can arise from the treatments and which profiles are exposed at severe risks.

\subparagraph*{Previous approach.}
The model synthesised in~\cite{DBLP:conf/cmsb/BowlesBBFGM24} has been used to synthesize the patient profiles that are more at risk, as a support for dynamic guideline revision: by refining guarded context to prevent severe effects for specific patients, we can readily check the efficacy of the changes.

\subparagraph*{GROOVE experimentation.}
TO BE COMPLETED



