% !TEX root =  ./main.tex

\section{Conclusion and Future Work}\label{sec:conc}


\textcolor{red}{
Flexibility of encoding:
\begin{itemize}
\item We can modify our system to other RS variants
\item Do another (small) case study? Durations?
\end{itemize}
}

In this work, we have demonstrated how Reaction Systems can be effectively encoded and analyzed within the \GROOVE framework, so to reuse the expressiveness and efficiency of graph transformation techniques. By exploiting quantified rules, the encoding consists of a direct translation from RS specification to a typed graph, which is made automatic in \BioResolve.
Then, \GROOVE enables both exhaustive state space exploration, the extraction of causal information through occurrence graphs and property-based verification based on model checking.

We have used \GROOVE to revisit several case studies from the literature and our experimental results, although preliminary, are promising: \GROOVE not only supports complex RS features such as guarded, nondeterministic and recursive contexts but also significantly improves performance and flexibility compared to existing RS tools.
Moreover, the use of \GROOVE’s recipes and model-checking capabilities opens the door to sophisticated analyses that were previously impractical.

A number of interesting avenues for future research remain open, among which we mention the possibility to extend the methodology to support quantitative RS variants with durations or weights~\cite{}; investigate alternative notions of causality and their representation in graph-based semantics, like dependencies drawn from inhibitors rather than reactants~\footnote{}; apply the \GROOVE toolset to further biological case studies, like those available in the CellCollective public repository~\cite{helikar2012cell}, possibly exploiting an automated pipeline for analysis.

Overall, the results show that graph transformation, and \GROOVE in particular, provide a robust and scalable foundation for the specification, execution, and analysis of Reaction Systems.