\documentclass[sn-mathphys-num,a4paper,iicol,lineno,pdflatex]{sn-jnl-hacked}
%%% RB: Hacking style to resemble camera ready layout of main text

% !TEX root =  ./main.tex


%%% RB additional packages
\usepackage[english]{babel}
\usepackage[T1]{fontenc}
\usepackage{comment}
\usepackage{hyphenat}
\usepackage{stmaryrd}
%\usepackage{wasysym}
\usepackage{proof} 
\usepackage{bussproofs}
 \usepackage[all]{xy}
\usepackage{mathtools} 
%\usepackage{lscape}
%\usepackage{cancel}
\usepackage{xspace}
\usepackage{subcaption}
\usepackage[textsize=tiny]{todonotes}
% Load hyperref (and then cleveref) last
\usepackage{hyperref}
\usepackage{cleveref}

%\usepackage[inline]{enumitem}
\newcommand{\lab}[1]{\textsf{#1}}
\newcommand{\blab}[1]{\lab{\bfseries #1}}
\newcommand{\ilab}[1]{\lab{\itshape #1}}

\newcommand{\nil}{\mathbf{0}}
\newcommand{\obs}[2]{\langle #1\vartriangleright #2\rangle}
\newcommand{\ccoin}{\mathsf{ccoin}}
\newcommand{\tcoin}{\mathsf{tcoin}}
\newcommand{\cpowder}{\mathsf{cpowder}}
\newcommand{\tpowder}{\mathsf{tpowder}}
\newcommand{\nomilk}{\mathsf{nomilk}}
\newcommand{\cappuccino}{\mathsf{cappuccino}}
\newcommand{\espresso}{\mathsf{espresso}}
\newcommand{\tea}{\mathsf{tea}}
\newcommand{\idle}{\mathsf{idle}}
\newcommand{\am}{\mathsf{am}}
\newcommand{\anger}{\mathsf{anger}}
\newcommand{\Forbidden}{\mathsf{Forbidden}}

\newcommand{\BioResolve}{\textsf{BioResolve}\xspace}
\newcommand{\GROOVE}{\textsf{GROOVE}\xspace}

% Node types
\newcommand{\Reaction}{{\text{\bfseries\sffamily Reaction}}\xspace}
\newcommand{\Rule}{{\text{\bfseries\sffamily\itshape Rule}}\xspace}
\newcommand{\Step}{{\text{\bfseries\sffamily Step}}\xspace}
\newcommand{\Entity}{{\text{\bfseries\sffamily Entity}}\xspace}
\newcommand{\State}{{\text{\bfseries\sffamily State}}\xspace}
\newcommand{\Token}{{\text{\bfseries\sffamily Token}}\xspace}
% Edges
\newcommand{\inhibitor}{{\text{\sffamily inhibitor}}\xspace}
\newcommand{\product}{{\text{\sffamily product}}\xspace}
\newcommand{\reactant}{{\text{\sffamily reactant}}\xspace}
\newcommand{\present}{{\text{\sffamily\itshape present}}\xspace}
\newcommand{\nextt}{{\text{\sffamily next}}\xspace}
\newcommand{\move}{{\text{\sffamily move}}\xspace}
% Flags
\newcommand{\fired}{{\text{\sffamily\itshape fired}}\xspace}

%Version 3 October 2023
% See section 11 of the User Manual for version history
%
%%%%%%%%%%%%%%%%%%%%%%%%%%%%%%%%%%%%%%%%%%%%%%%%%%%%%%%%%%%%%%%%%%%%%%
%%                                                                 %%
%% Please do not use \input{...} to include other tex files.       %%
%% Submit your LaTeX manuscript as one .tex document.              %%
%%                                                                 %%
%% All additional figures and files should be attached             %%
%% separately and not embedded in the \TeX\ document itself.       %%
%%                                                                 %%
%%%%%%%%%%%%%%%%%%%%%%%%%%%%%%%%%%%%%%%%%%%%%%%%%%%%%%%%%%%%%%%%%%%%%

%%\documentclass[referee,sn-basic]{sn-jnl}% referee option is meant for double line spacing

%%=======================================================%%
%% to print line numbers in the margin use lineno option %%
%%=======================================================%%

%%\documentclass[lineno,sn-basic]{sn-jnl}% Basic Springer Nature Reference Style/Chemistry Reference Style

%%======================================================%%
%% to compile with pdflatex/xelatex use pdflatex option %%
%%======================================================%%

%%\documentclass[pdflatex,sn-basic]{sn-jnl}% Basic Springer Nature Reference Style/Chemistry Reference Style

%%Note: the following reference styles support Namedate and Numbered referencing. By default the style follows the most common style. To switch between the options you can add or remove “Numbered” in the optional parenthesis. 
%%The option is available for: sn-basic.bst, sn-vancouver.bst, sn-chicago.bst%  
 
%%\documentclass[sn-nature]{sn-jnl}% Style for submissions to Nature Portfolio journals
%%\documentclass[sn-basic]{sn-jnl}% Basic Springer Nature Reference Style/Chemistry Reference Style
%%\documentclass[sn-mathphys-num]{sn-jnl}% Math and Physical Sciences Numbered Reference Style 
%%\documentclass[sn-mathphys-ay]{sn-jnl}% Math and Physical Sciences Author Year Reference Style
%%\documentclass[sn-aps]{sn-jnl}% American Physical Society (APS) Reference Style
%%\documentclass[sn-vancouver,Numbered]{sn-jnl}% Vancouver Reference Style
%%\documentclass[sn-apa]{sn-jnl}% APA Reference Style 
%%\documentclass[sn-chicago]{sn-jnl}% Chicago-based Humanities Reference Style

%%%% Standard Packages
%%<additional latex packages if required can be included here>

\usepackage{graphicx}%
\usepackage{multirow}%
\usepackage{amsmath,amssymb,amsfonts}%
\usepackage{amsthm}%
\usepackage{mathrsfs}%
\usepackage[title]{appendix}%
\usepackage{xcolor}%
\usepackage{textcomp}%
\usepackage{manyfoot}%
\usepackage{booktabs}%
\usepackage{algorithm}%
\usepackage{algorithmicx}%
\usepackage{algpseudocode}%
\usepackage{listings}%
%%%%

%%%%%=============================================================================%%%%
%%%%  Remarks: This template is provided to aid authors with the preparation
%%%%  of original research articles intended for submission to journals published 
%%%%  by Springer Nature. The guidance has been prepared in partnership with 
%%%%  production teams to conform to Springer Nature technical requirements. 
%%%%  Editorial and presentation requirements differ among journal portfolios and 
%%%%  research disciplines. You may find sections in this template are irrelevant 
%%%%  to your work and are empowered to omit any such section if allowed by the 
%%%%  journal you intend to submit to. The submission guidelines and policies 
%%%%  of the journal take precedence. A detailed User Manual is available in the 
%%%%  template package for technical guidance.
%%%%%=============================================================================%%%%

%% as per the requirement new theorem styles can be included as shown below
\theoremstyle{thmstyleone}%
\newtheorem{theorem}{Theorem}%  meant for continuous numbers
%%\newtheorem{theorem}{Theorem}[section]% meant for sectionwise numbers
%% optional argument [theorem] produces theorem numbering sequence instead of independent numbers for Proposition
\newtheorem{proposition}[theorem]{Proposition}% 
%%\newtheorem{proposition}{Proposition}% to get separate numbers for theorem and proposition etc.

\theoremstyle{thmstyletwo}%
\newtheorem{example}{Example}%
\newtheorem{remark}{Remark}%

\theoremstyle{thmstylethree}%
\newtheorem{definition}{Definition}%

\raggedbottom
%%\unnumbered% uncomment this for unnumbered level heads

%%%%%% To display ORCID Logo with link, Please add below definition and copy the ORCID_Color.eps in the manuscript package %%%%%
\makeatletter
	\def\@citecolor{blue}%
	\def\@urlcolor{blue}%
	\def\@linkcolor{blue}%
	\def\UrlFont{\rmfamily}
	\def\orcidID#1{\href{http://orcid.org/#1}{\includegraphics{ORCID_Color.eps}}}
\makeatother

\begin{document}

\title[Experimenting with Reaction Systems using Graph Transformation]{Experimenting with Reaction Systems using Graph Transformation}

%%=============================================================%%
%% GivenName	-> \fnm{Joergen W.}
%% Particle	-> \spfx{van der} -> surname prefix
%% FamilyName	-> \sur{Ploeg}
%% Suffix	-> \sfx{IV}
%% \author*[1,2]{\fnm{Joergen W.} \spfx{van der} \sur{Ploeg} 
%%  \sfx{IV}}\email{iauthor@gmail.com}
%%=============================================================%%

\author[1]{\fnm{Roberto} \sur{Bruni\orcidID{0000-0002-7771-4154}}}\email{roberto.bruni@unipi.it}
%\equalcont{These authors contributed equally to this work.}

\author*[2]{\fnm{Arend} \sur{Rensink\orcidID{0000-0002-1714-6319}}}\email{arend.rensink@utwente.nl}
%\equalcont{These authors contributed equally to this work.}

\affil[1]{\orgdiv{CS Dept.}, \orgname{University of Pisa}, \orgaddress{\street{Largo B.\ Pontecorvo, 3}, \city{Pisa}, \postcode{56127},  \country{Italy}}}

\affil*[2]{\orgdiv{CS Dept.}, \orgname{University of Twente}, \orgaddress{\street{Hallenweg 19}, \city{Enschede}, \postcode{7522}, \country{Netherlands}}}

\abstract{We explore the capabilities of GROOVE, a state-of-the-art toolset based on graph transformation systems, to perform reachability and causal analysis of Reaction Systems.
Our results are still preliminary, but encouraging, as in the presence of large state spaces GROOVE outperforms available applications by halving the analysis time of both reachability and causal analyses.
From the point of view of GROOVE, the implementation of Reaction Systems provided some interesting insights on the most convenient way to model certain computational requirements through negative and nested application conditions.}

\keywords{Reaction Systems, Graph Transformation, GROOVE, \textcolor{red}{Other keywords}}

%%\pacs[JEL Classification]{D8, H51}

%%\pacs[MSC Classification]{35A01, 65L10, 65L12, 65L20, 65L70}

\maketitle

% !TEX root =  ./main.tex

\section{Introduction}

We explore the capabilities of a state-of-the-art toolset based on graph transformation to perform reachability, causal analysis and verification of complex systems modelled using Reaction Systems.

Reaction systems (RS)~\cite{DBLP:journals/fuin/EhrenfeuchtR07} are a computational model inspired by the functioning of biochemical reactions within living cells. 
%The primary motivation behind RS is to provide a simple yet expressive model for understanding and analyzing processes in natural and artificial systems.
RS focus on the interaction of entities through a set of reactions. 
Each reaction relies on some reactants, inhibitors, and products to mimic two fundamental mechanisms found in nature: facilitation and inhibition.
%Facilitation means that a reaction can occur only if all of its reactants are present, while inhibition means that a reaction cannot occur if any of its inhibitors is present. 
At each time instant, the next state of the system is determined by the products of all enabled reactions plus some additional entities that are possibly provided by environment.
Unlike traditional models of concurrency, like Petri nets, the theory of RS is based on three principles: \emph{no permanency}, any entity vanishes unless it is sustained by a reaction; \emph{no competition}, an entity is either available for all reactions, or it is not available at all; and \emph{no counting}, the exact concentration level of available entities is ignored.
Moreover, due to the use of inhibitors, RS can exhibit non monotonic behaviour, in the sense that what can be done with less resources is not necessarily replicable with more resources.
Since their introduction, RS have been successfully applied to the analysis of complex systems in many different fields~\cite{ABP14,CMMBM12,Az17,OY16,DBLP:journals/ijfcs/EhrenfeuchtMR10,DBLP:journals/ijfcs/EhrenfeuchtMR11}.
Recent applications concerned, e.g., with the efficacy of medical treatments for comorbidities and with the selection of the best environment to achieve some desired phenomena~\cite{??}, led to experimenting with environments that exhibit nondeterministic and recursive behaviour.
Then, performing reachability and causal analysis required the exploration of large state spaces for which the available prototype tool~\cite{DBLP:journals/tcs/BrodoBF21} struggled in terms of  memory consumption and response time.
 
Graph Transformation (GT)~\cite{DBLP:series/eatcs/EhrigEPT06,DBLP:books/sp/HeckelT20} is a modelling technique that is widely applicable in problem domains where the objects of study have an inherent graphical structure, and the task at hand is to study their properties and evolution. Besides the graphs themselves, the core concept is that of a (transformation) \emph{rule} capturing a particular change to such a graph. Rules can be used, for instance, to describe the change of a system over time, but can also be instrumental in composing and decomposing graphs and so exposing structural properties.
Since RS can be derived from Boolean network models and visualized themselves as suitable networks of reactions, it is quite natural trying to embed them within the GT framework to take advantage of well established analysis techniques.

Importantly, from a practical point of view, there are a number of (academic) tools supporting the use of GT. The research described here crucially relies on \href{https://groove.cs.utwente.nl}{GROOVE}~\cite{DBLP:journals/sttt/GhamarianMRZZ12}, one of the most prominent tools in this area, which was designed precisely to enable GT-based system analysis of the kind described above. The features of GROOVE that are essential for the purpose of this research are
\begin{enumerate}%[label=\emph{(\roman*)}]
\item nested (i.e., quantified) rules, which capture simultaneous changes in all neighbourhoods that satisfy certain application conditions, rather than only locally in one such neighbourhood at a time; 
\item exploration of the set of reachable graphs (under the given rules) using various strategies;
\item model checking functionalities that can be used to validate previous findings as well to explore and support the study of new modal properties.
\end{enumerate}

The main research question that motivated our study is: 
\emph{how can GT help in addressing the analysis of Reaction Systems?} 
For this purpose, we encode a given RS as a single graph, upon which a small number of (predefined) rules operate to simulate the correct semantics. The core rule describes the simultaneous firing of all \blab{Reaction}s of which no \lab{inhibitor}s and all \lab{reactant}s are present; the firing results in the presence of all \lab{product}s. Simultaneously, all currently present \blab{Entity}s are removed. In GROOVE syntax, this looks as follows:

\begin{center}
\includegraphics[scale=.40]{react}
\end{center}

To parse this, note that (in the GROOVE rule syntax) the red structure must be absent for the rule to apply; moreover, green labels are added and blue ones deleted upon rule application. The \ilab{present} flag signals whether an \blab{Entity} is considered to be currently present; hence, creating or deleting that flag comes down to creating or deleting the \blab{Entity}. The $\forall$-nodes impose the desired quantification, causing a single application of this rule to model the firing of all enabled \blab{Reaction}s, even if there are thousands of them.

Our model also supports environments that inject \blab{Entity}s in a controlled manner. This is achieved by encoding the context specification in the initial graph and exploiting a second predefined rule, not shown here. A configuration is reachable if it can be constructed by the alternating application of both rules: First 

\medskip\noindent
Our first results are encouraging: GROOVE seems capable to explore larger Reaction Systems 
%is well beyond what other tools have been able to achieve, 
by halving the analysis time of both reachability and causal analyses.
More precisely, we are able to find a trace towards unwanted patterns (if they exist) among millions of reachable configurations using different heuristics; then, we can also prune the trace to extract a graphical representation of the causal history of that entities, at least as far as reactants are concerned.

Action Points:
\begin{itemize}
\item Develop toy example
\item Think about possible extensions
\end{itemize}




\section{Background}
% !TEX root =  ./main.tex

\subsection{RSs with Guarded Contexts}\label{sec:RS}

First, we briefly account for the classical set theoretic definition of Reaction Systems (RSs)~\cite{DBLP:journals/fuin/EhrenfeuchtR07}. Then, we focus on their process algebraic version~\cite{DBLP:journals/tcs/BrodoBF21} and its further extension with guarded contexts in~\cite{DBLP:conf/cmsb/BowlesBBFGM24}. 


\subparagraph*{RS basics.}
A Reaction System is a pair ${\cal A} = (S, A)$, where $S$ is the finite set of \emph{entities}, and $A$ is a finite set of \emph{reactions} of the form $a = (R,I,P)$, with $R, I, P\subseteq S$ and $R \cap I = \emptyset$. 
The sets $R, I, P$ are the sets of \emph{reactants}, \emph{inhibitors}, and  \emph{products}, respectively. 
Without loss of generality, we admit the use of empty sets as reactants or inhibitors.
%
Given the current state $W\subseteq S$, a reaction $a = (R,I,P)$ is enabled in $W$ if all its reactants are present (i.e., $R\subseteq W$) and all its inhibitors absent (i.e., $W \cap I = \emptyset$).
The \emph{result} of the reaction $a$ on the current state $W$ is $P$ if $a$ is enabled, and
$\emptyset$ otherwise.
The result of all reactions $A$ on the current state $W$, is the union of the results of all reactions.
%
The no-permanency principle of RSs dictates that entities disappear if not sustained by some reaction.
Thus, the current state $W=D\cup C$ is determined by the result $D$ of all reactions on the previous state, together with some additional entities $C$ that can be provided by the \emph{context} at each step. 

\subparagraph*{Process algebraic RSs and guarded contexts.}
Inspired by Plotkin's Structural Operational Semantics approach~\cite{DBLP:journals/jlp/Plotkin04a} and process algebras such as CCS~\cite{Milner80}, the key features of the process algebraic version of RSs are compositionality and the ability to account for a quite general notion of context (guarded, nondeterministic, recursive) using a friendly syntax. This way, we derive a Labelled Transition System (LTS) semantics for RSs by means of inductive inference rules, where LTS states are terms of an algebra, each transition defines a computation step of the RS and its label records the entities involved in that step.

\begin{definition}[RS processes]\label{def:LTSforRS}
%Let $S$ be a set of entities. 
\emph{RS processes} are defined by the grammar below:

\begin{eqnarray*}
\mathsf{P} & := & [\mathsf{M}]
\\
\mathsf{M} & := & (R,I,P) \mid D \mid \mathsf{K} \mid \mathsf{M}|\mathsf{M}
\\
\mathsf{K} & ::= & \nil \mid (R,I,C).\mathsf{K} \mid \mathsf{K}+\mathsf{K} \mid X
\end{eqnarray*}

\noindent
where $R$, $I$, $P$, $C$, and $D$ are sets of entities (with $P\neq \emptyset$ and $R\cap I=\emptyset$) and $X$ is a context identifier drawn from a family of (recursive) definitions $\Delta \triangleq\{X_j=\mathsf{K}_j\}_{j\in J}$, called the \emph{environment}.
\end{definition}

Roughly, a RS process  $\mathsf{P}$ embeds a \emph{mixture} process $\mathsf{M}$ obtained as the parallel composition of some reactions $(R,I,P)$, some available entities $D$ (if any), and some \emph{context} process $\mathsf{K}$.
%We write $\prod_{i\in I} \mathsf{M}_i$ for the parallel composition of all $\mathsf{M}_i$ with $i\in I$. 
A  context process $\mathsf{K}$ is either: 
the nil context $\nil$ that stops the computation;
the guarded context $(R,I,C).\mathsf{K}$ that makes the entities $C$ available to the reactions if the reactants $R$ are present and the inhibitors $I$ are absent, and then will behave as $\mathsf{K}$ at the next step;
the non-deterministic choice $\mathsf{K}_1+\mathsf{K}_2$ that can behave as either  $\mathsf{K}_1$ or $\mathsf{K}_2$;  
the context identifier $X$ that behaves as $\mathsf{K}$ for $X=\mathsf{K}\in \Delta$.
We write $C.\mathsf{K}$ as a shorthand for the trivially guarded process $(\emptyset,\emptyset,C).\mathsf{K}$ and we assume the recursive context $\mathsf{Emp}=\emptyset.\mathsf{Emp}$ is always defined.


We say that $\mathsf{P}$ and $\mathsf{P}'$ are structurally equivalent, written $\mathsf{P} \equiv \mathsf{P}'$, when they denote the same term up to the laws of Abelian monoids (unit, associativity and commutativity) for  parallel composition $\cdot | \cdot$, with $\emptyset$ as the unit, and the laws of idempotent Abelian monoids for choice $\cdot +\cdot$, with $\nil$ as the unit. We also assume $D_1 | D_2 \equiv D_1\cup D_2$ for any $D_1,D_2\subseteq S$.
Indexed sums and parallel compositions are denoted, respectively, by $\sum_{i\in I} \mathsf{K}_i$ and $\prod_{i\in I} \mathsf{M}_i$.

The SOS semantics of  RS processes is defined by the SOS rules in Fig.~\ref{fig:guardforRS2nd}.
A transition label $\ell$, written $\obs{\obs{D}{R',I',C}}{R,I,P}$, records:
the available entities $D$; the entities $C$ provided by the guarded contexts, assuming all entities in $R'$ are present and those in $I'$ are absent;
the set $R$ of entities whose presence enables or disables some reactions;
the set $I$ of entities whose absence  enables or disables some reactions;
and the set $P$ of reaction products.
The  rules guarantee that, whenever $\mathsf{P}\xrightarrow{\obs{\obs{D}{R',I',C}}{R,I,P}} \mathsf{P}'$, it holds that $(R',I',C)$ is enabled in $D$ and that
$(R,I,P)$ is enabled in $W\triangleq (D\cup C)$.

\begin{figure*}[t]
		$$  
		\infer[(\textit{\scriptsize{Ent}})]
		{D \xrightarrow{\obs{\obs{D}{\emptyset,\emptyset,\emptyset}}{\emptyset,\emptyset,\emptyset}}\emptyset}
		{}
		\qquad
		\infer[(\textit{\scriptsize{Cxt}})]
		{(R,I,C).\mathsf{K} \xrightarrow{\obs{\obs{\emptyset}{R,I,C}}{\emptyset,\emptyset,\emptyset}}\mathsf{K}}{}
		$$
		$$
		\infer[(\textit{\scriptsize Suml})]
		{\mathsf{K}_1 + \mathsf{K}_2 \xrightarrow{\ell}\mathsf{K}'_1}
		{\mathsf{K}_1 \xrightarrow{\ell}\mathsf{K}'_1}
		\qquad
		\infer[(\textit{\scriptsize Sumr})]
		{\mathsf{K}_1 + \mathsf{K}_2 \xrightarrow{\ell}\mathsf{K}'_2}
		{\mathsf{K}_2 \xrightarrow{\ell}\mathsf{K}'_2}
		\qquad
		\infer[(\textit{\scriptsize Rec})]
		{X \xrightarrow{\ell}\mathsf{K}'}
		{X=\mathsf{K}\in\Delta & \mathsf{K} \xrightarrow{\ell}\mathsf{K}'}
		$$
		$$
		\infer[(\textit{\scriptsize Pro})]
		{(R,I,P)  \xrightarrow{\obs{\obs{\emptyset}{\emptyset,\emptyset,\emptyset}}{R,I,P}}(R,I,P)|P}
		{}
		\qquad
		\infer[(\textit{\scriptsize Inh})]
		{(R,I,P)  \xrightarrow{\obs{\obs{\emptyset}{\emptyset,\emptyset,\emptyset}}{J,Q,\emptyset}}(R,I,P)}
		{J \subseteq I & Q \subseteq R & J\cup Q\neq \emptyset}
		$$
		$$
		\infer[(\textit{\scriptsize Par})]
		{\mathsf{M}_1~|~\mathsf{M}_2\xrightarrow{\ell_1\cup\ell_2} \mathsf{M}'_1~|~\mathsf{M}'_2}
		{\mathsf{M}_1 \xrightarrow{\ell_1} \mathsf{M}'_1 &
		\mathsf{M}_2 \xrightarrow{\ell_2} \mathsf{M}'_2 &
			\ell_1\frown \ell_2 }
		\qquad
		\infer[(\textit{\scriptsize Sys})]
		{[\mathsf{M}]\xrightarrow{\obs{\obs{D}{R',I',C}}{R,I,P}} [\mathsf{M}']}
		{\mathsf{M}\xrightarrow{\obs{\obs{D}{R',I',C}}{R,I,P}} \mathsf{M}' &
		R'\subseteq D &
        R\subseteq D\cup C}
		$$

\medskip	
\noindent
{\footnotesize
		where $\ell_1 \frown \ell_2$ and $\ell_1 \cup \ell_2$ are defined as follows:
		$$\begin{array}{l}
\obs{\obs{D_1}{R'_1,I'_1,C_1}}{R_1,I_1,P_1}
\frown
\obs{\obs{D_2}{R'_2,I'_2,C_2}}{R_2,I_2,P_2}
\\
\qquad\qquad\qquad\qquad\qquad\qquad\qquad\qquad\qquad\qquad\qquad\qquad\qquad
\triangleq (\textstyle\bigcup_{i=1,2} D_i\cup  R'_i) \cap (I'_1 \cup I'_2) = \emptyset
\wedge
(\bigcup_{i=1,2} D_i\cup  C_i\cup  R_i) \cap (I_1 \cup I_2) = \emptyset \\[5pt]
\obs{\obs{D_1}{R'_1,I'_1,C_1}}{R_1,I_1,P_1}
\cup
\obs{\obs{D_2}{R'_2,I'_2,C_2}}{R_2,I_2,P_2}
\\
\qquad\qquad\qquad\qquad\qquad\qquad\qquad\qquad\qquad\qquad\qquad\qquad\qquad
\triangleq \obs{\obs{D_1\cup D_2}{R'_1\cup R'_2,I'_1\cup I'_2,C_1\cup C_2}}{R_1\cup R_2,I_1\cup I_2,P_1\cup P_2}
\end{array}$$
}
		\caption{SOS semantics of the RS processes.}
		\label{fig:guardforRS2nd}
\end{figure*}


The rule $(\textit{Ent})$ records the set of current entities $D$.
By rule $(\textit{Cxt})$, a guarded context process $(R,I,C).\mathsf{K}$ makes available the entities in $C$ if the reactants $R$ are present in the current state and the inhibitors $I$ are absent, and then reduces to $\mathsf{K}$. 
Rules $(\textit{Suml})$ and $(\textit{Sumr})$ select a move of either the left or the right context, resp., discarding the other process.
By rule $(\textit{Rec})$, a context identifier $X$ behaves according to its defining process $\mathsf{K}$.

The rule $(\textit{Pro})$ assumes the reaction $(R,I,P)$ is enabled: it records its reactants, inhibitors, and products in the label, and leaves the reaction  available at the next step, together with its products $P$.
The rule $(\textit{Inh})$ records in the label the reasons why the reaction $(R,I,P)$ should not be executed: possibly some inhibiting entities $(J \subseteq I)$ are present or some reactants $(Q \subseteq R)$ are missing, with $J \cup Q \neq \emptyset$, as at least one cause is needed.

The rule $(\textit{Par})$ puts two processes in parallel by pooling their labels and joining all labels components. We write $\ell_1\cup\ell_2$ for the component-wise union of labels, while the sanity check $\ell_1\frown\ell_2$ is required to guarantee that labels of reactants and inhibitors are consistent (see definitions in Fig.~\ref{fig:guardforRS2nd}).

Finally, the rule $(\textit{Sys})$ checks that all the needed reactants are available in the system. Checking the absence of inhibitors  is not necessary, thanks to the sanity check in rule $(\textit{Par})$.
Note that, while the enabling of $(R,I,C).\mathsf{K}$ requires the presence of reactants $R$ and the absence of inhibitors $I$ w.r.t. the set of current entities $D$, in the case of reactions $(R,I,P)$, the check is performed w.r.t. the current state $W=D\cup C$.
More importantly, the products $C$ are made available immediately from the context, not at the next step.
It is worthy to mention that a conditional prefixed process that is not enabled behaves as the $\nil$ process.

A first concrete example of RS that exposes most of the features is presented in Section~\ref{sec:student}.


% !TEX root =  ./main.tex

\subsection{GT and GROOVE}\label{sec:GTS}

Graph Transformation (GT, sometimes called Graph Rewriting) is a well-established rule-based formalism, the core of which is to specify precisely how graphs may evolve. Each rule embodies a particular change, which can be applied to a given graph (in the simplest form consisting of nodes and binary edges) by establishing where in that graph the preconditions of the rule are met, and then adding and deleting nodes and edges as prescribed.

In this paper, we use the so-called \emph{algebraic approach} to graph transformation (see \cite{DBLP:series/eatcs/EhrigEPT06} for a formal exposition and \cite{DBLP:books/sp/HeckelT20} for applications in the context of software engineering); moreover, we rely on the particular flavour implemented in the tool GROOVE \cite{DBLP:journals/sttt/GhamarianMRZZ12,GROOVE}. Some of the relevant features of the approach and the tool are highlighted below.
%\todo{RB: the previous itemized list has been just commented in the source .tex, not deleted}
%
%\begin{itemize}
%\item Graphs are \emph{directed} and \emph{typed}, meaning that nodes and edges have labels, and edges have a direction (going from their \emph{source} to their \emph{target}). Besides binary edges, nodes can also have \emph{flags} (which are actually self-loops that act as additional, optional labels on nodes) and \emph{attributes} (which are actually binary edges whose target node is a data value, e.g., an integer or string).
%
%\item Rules consist of a left hand side (LHS) and right hand side (RHS). Rule applicability is established by \emph{matching} the LHS to the graph in question, and where a match exists, removing nodes and edges that are in the LHS but not in the RHS, and vice versa, adding nodes and edges that are in the RHS but not in the LHS.
%
%\item \GROOVE graphs are \emph{simple} (rather than \emph{multi-sorted}), meaning that there is at most one edge of a given type between any two nodes, and similarly, at most one flag of a given type on any node. If an edge or flag is added where one exists already, this does not change the graph.
%
%\item \GROOVE rules may be \emph{quantified}, meaning that they can simultaneously be applied to multiple places in the same graph. An example is shown in \Cref{fig:context} below.
%
%\item A graph transformation \emph{system} is a set of rules. In the simplest case, every rule can be applied to a graph at hand, giving rise to a modified graph to which every rule can be applied again, and so forth. However, \GROOVE allows rule application to be restricted by an external \emph{control program} that can specify in what order rules may be applied.
%
%\item The core functionality of \GROOVE is to explore the set of reachable graphs, given a graph transformation system (with or without control program) and an initial graph. With this as the basis, there are built-in strategies for searching and model checking.
%\end{itemize}
%

\textbf{Graphs} are \emph{simple} and \emph{typed}, meaning that there is at most one edge of a given type between any two nodes and that all nodes and edges are labelled through a morphism to a given (fixed) \emph{type graph}. Edges are \emph{directed} (going from their \emph{source} to their \emph{target}). Besides binary edges, nodes can also have \emph{flags} (which are actually self-loops that act as additional, optional labels on nodes) and \emph{attributes} (which are actually binary edges whose target node is a data value, e.g., an integer or a string).

\textbf{Rules}, in their simplest form, consist of a left hand side (LHS) and right hand side (RHS). Rule applicability is established by \emph{matching} the LHS to the graph in question, and where a match exists, removing nodes and edges that are in the LHS but not in the RHS, and vice versa, adding nodes and edges that are in the RHS but not in the LHS. In addition, however, \GROOVE supports \emph{quantified} rules, which can simultaneously be applied to multiple places in the same graph. An example is shown in \Cref{fig:context} below.

\textbf{Evolution} of a graph is defined on the basis of a graph transformation \emph{system}, which is a set of rules applied to a graph at hand, giving rise to a modified graph to which every rule can be applied again, and so forth. On top of this, \GROOVE allows for \emph{control programs} that can specify in what order rules may be applied. By exploring the potential evolution of a graph in all ways allowed by the control program, \GROOVE constructs the \emph{state space} of the graph transformation system, in the form of a \emph{labelled transition system} consisting of all reachable graphs and the rule applications between them.

\textbf{Analysis} consists of the exploration of the state space for a given initial graph, rule system and (optional) control program. The exploration can be tuned by built-in strategies for searching and model checking.
%\todo{RB: slightly rearranged}
%The core functionality of \GROOVE is to explore the set of reachable graphs, given a graph transformation system (with or without control program) and an initial graph. With this as the basis, there are built-in strategies for searching and model checking.

\smallskip\noindent
The power of graph transformation lies in its generality: many systems naturally lend themselves to be modelled as graphs, and algebraic rules --- especially quantified ones --- provide a rich framework to specify their evolution. This is in fact our motivation for using it the current paper: reaction systems can straightforwardly be interpreted as graphs. \Cref{fig:core-type} shows the core types for the relevant concepts of that interpretation. (The colours just support the visualisation and have no semantics of their own.)

\begin{figure}
\centering
\includegraphics[scale=.5]{figs/core-type}
\caption{Core type graph for reaction systems}
\label{fig:core-type}
\end{figure}
%
Note especially the (abstract) supertype \Rule with subtypes \Reaction and \Step: the former is the type for the elements of $A$ in a Reaction System ${\cal A}=(S,A)$, whereas the latter is used to represent triples $(R,I,P)$ in a context RS process (\Cref{def:LTSforRS}). The flag \fired is used to mark \Rule{}s that have triggered in the most recent step. The set $S$ is represented by nodes of type \Entity; for a given \Rule, the subsets $R$, $I$, and $P$ of $S$ are those \Entity{}s to which there is an outgoing edge labelled \reactant, \inhibitor or \product. The subtype \Forbidden anticipates the principle, demonstrated later in this paper, of identifying undersirable entities and specifically searching for scenarios in which those are produced. The flag \present is used to label the entities occurring in a state $W$. Finally, the structure of (guarded) context processes is captured by \State entities, with \nextt-edges to the \Step{}s that can be nondeterministally chosen; the subsequent process after such a \Step is determined by its outgoing \move-edge. \Token nodes are used to model which \State{}s are currently active.



% !TEX root =  ./main.tex

\section{Toy Example}

Coffee machine? ATM?


% !TEX root =  ./main.tex


\section{Encoding of RS in GT}
\begin{itemize}
\item Rules for implementing RS semantics
\item Conversion of a trace to a control program
\item Using recipes in the occurrence graph building
\end{itemize}


% !TEX root =  ./main.tex

\section{Experimentation}\label{sec:experiments}

Here we consider 
three larger case studies whose RS specifications have already appeared in the recent literature. For each case study, we briefly describe its main features and then show how the methodology outlined in the previous sections can be applied for carrying out some fruitful experimentation with \GROOVE.

All \GROOVE experiments were carried out on a Dell Precision 3551 laptop with an Intel i7 CPU running at 2.6 GHz; \GROOVE was run in a Java~24 JVM with 12GB of memory. No attempt was made to measure running time with precision, and repeated experiments have shown that the reported durations can deviate up to 10\%.


% !TEX root =  ./main.tex

\section{Comparison}
(\url{https://www.reactionsystems.org/about-reaction-systems})
\begin{itemize}
\item BioReSolve
\item Heresy
\item WebRSim
\item CL-RS (\url{https://github.com/mnzluca/cl-rs})
\item Maude
\end{itemize}


% !TEX root =  ./main.tex

\section{Conclusion and Future Work}\label{sec:conc}


%\textcolor{red}{
%Flexibility of encoding:
%\begin{itemize}
%\item We can modify our system to other RS variants
%\item Do another (small) case study? Durations?
%\end{itemize}
%}

In this work, we have demonstrated how Reaction Systems can be effectively encoded and analyzed within the \GROOVE framework, so to reuse the expressiveness and efficiency of graph transformation techniques. By exploiting quantified rules, the encoding consists of a direct translation from RS specification to a typed graph, which is made automatic in \BioResolve.
Then, \GROOVE enables both exhaustive state space exploration, the extraction of causal information through occurrence graphs and property-based verification based on model checking.

We have used \GROOVE to revisit several case studies from the literature and our experimental results, although preliminary, are promising: \GROOVE not only supports complex RS features such as guarded, nondeterministic and recursive contexts but also significantly improves performance and flexibility compared to existing RS tools.
Moreover, the use of \GROOVE’s recipes and model-checking capabilities opens the door to sophisticated analyses that were previously impractical.

\todo{RB: should we mention the possibility to improve the usability of GROOVE for non experts by the preparation of dedicated shell scripts for different kinds of analyses?}
A number of interesting avenues for future research remain open, among which we mention the possibility to extend the methodology to support quantitative RS variants with durations or weights~\cite{DBLP:journals/nca/BrodoBFGLM23}; investigate alternative notions of causality and their representation in graph-based semantics, like dependencies drawn from inhibitors rather than reactants;~\footnote{In this respect, we could, e.g., exploit the semantic-preserving transformation from RSs to Positive RSs proposed in~\cite{DBLP:journals/sttt/BrodoBFGMMP24}.} apply the \GROOVE toolset to further biological case studies, like those available in the CellCollective public repository~\cite{helikar2012cell}, possibly exploiting an automated pipeline for analysis.

Overall, the results show that graph transformation, and \GROOVE in particular, provide a robust and scalable foundation for the specification, execution, and analysis of Reaction Systems.

\backmatter

\bmhead{Supplementary information}

If your article has accompanying supplementary file/s please state so here. 

\bmhead{Acknowledgements}

\section*{Declarations}

\subsection*{Funding}

Research partially supported 
by \textcolor{red}{Other projects}
by the PRIN PNRR 2022 project \emph{Resource Awareness in Programming} (RAP, P2022HXNSC),
by the University of Pisa PRA project \emph{Formal methods for the healthcare domain based on spatial information} (FM4HD, PRA\_2022\_99),
and by the INdAM GNCS project CUP\_E53C22001930001.

\subsection*{Ethical Approval}
This is not applicable.
 
\subsection*{Competing interests}
The authors have no relevant financial or non-financial interests to disclose.

\subsection*{Availability of data and materials}
Data Availability Statement: No Data associated in the manuscript.

\subsection*{Code availability}

\textcolor{red}{Pointer to git repository?}

\subsection*{Author contribution}

All co-authors contributed equally to this work.

\begin{appendices}

% !TEX root =  ./main.tex

\section{Auxiliary Material}

\subsection{Auxiliary Material for the toy running example}\label{app:running}

The \BioResolve specification for the toy running example about the interaction between the student and the vending machine is reported in \Cref{fig:bioresolve:toy}. The corresponding RS has been described in \Cref{sec:student} and it has been used to illustrate some key features of the \GROOVE encoding in \Cref{sec:RS2GTS}.

\begin{figure}[t]
\begin{minipage}{0.9\linewidth}
\footnotesize
\begin{verbatim}
myentities([cpowder,tpowder]). % initial set D0

myreactions([                  % list of reactions
    react([idle],[am],[am]),
    react([am],[idle],[am]),
    react([ccoin,cpowder],[nomilk],[cappuccino]),
    react([ccoin,cpowder,nomilk],[],[espresso]),
    react([tcoin,tpowder],[],[tea]),
    react([cpowder],[],[cpowder]),
    react([tpowder],[],[tpowder]),
    react([anger],[],[bang]) ]).

mycontext("[refill,student]"). % context processes

myenvironment("[               % context definitions
    refill = ({nomilk}.refill 
            + {}.refill),
    student = (?{},{am},{tcoin}?.gettea 
             + ?{am},{},{ccoin}?.getcappuccino 
             + {idle}.student),
    gettea = (?{tea},{},{}?.student
            + ?{},{tea},{anger}?.student),
    getcappuccino = (?{cappuccino},{},{}?.student 
                    + ?{espresso},{},{anger}?.student) ]").
\end{verbatim}
\end{minipage}
\caption{\BioResolve implementation of the vending machine RS from \Cref{sec:student}. The question marks \texttt{?} are used to delimit guarded prefixes in context processes.}
\label{fig:bioresolve:toy}
\end{figure}

\subsection{Auxiliary Material for the Comorbidity Case Study}

\newcommand{\ent}[1]{\mathsf{#1}}
\newcommand{\ents}[2]{\mathsf{#1}\texttt{\_}\mathsf{#2}}

The RS specification for the comorbidity case study presented in~\cite{DBLP:conf/cmsb/BowlesBBFGM24} is reported in \Cref{fig:bioresolve:comorbidities:reactions} (set of reactions) and \Cref{fig:bioresolve:comorbidities:contexts} (context processes definitions), where we assume the initial state is $\mathsf{D}_0  =  \varnothing$.
The corresponding experimentation with \GROOVE has been discussed in \Cref{sec:cmsb2024}.

\begin{figure*}[t]
\fontsize{6}{0}
\[
\begin{array}{rcl}
\mathsf{Feats} &  \triangleq 
& (\{\ent{hyper}\},\varnothing,\{\ent{hyper}\})
\mid  (\{\mathsf{afib}\},\varnothing,\{\ent{afib}\})
\mid  (\{\ents{has}{fib}\},\varnothing,\{\ents{has}{fib}\})
\mid  (\{\ents{heart}{rate}\},\varnothing,\{\ents{heart}{rate}\})
\mid  (\{\ents{consensus}{acei}\},\varnothing,\{\ents{consensus}{acei}\})
\\[-4pt] & \mid &  (\{\ent{over75}\},\varnothing,\{\ent{over75}\})
\mid  (\{\ent{below55}\},\varnothing,\{\ent{below55}\})
\mid  (\{\ent{diabete}\},\varnothing,\{\ent{diabete}\})
\mid  (\{\ent{origin}\},\varnothing,\{\ent{origin}\})
\\[-4pt] & \mid &  (\{\ents{doac}{int}\},\varnothing,\{\ents{doac}{int}\})
\mid  (\{\ent{hyper}\},\varnothing,\{\ent{diseases}\})
\mid  (\{\ent{diabete}\},\varnothing,\{\ent{diseases}\})
\\[-4pt]
\mathsf{Drugs} &  \triangleq 
& (\{\ents{get}{diltiazem}\},\{\ents{stop}{cbb}\},\{\ent{diltiazem},\ent{cbb}\})
\mid  (\{\ent{diltiazem}\},\{\ents{stop}{cbb}\},\{\ent{diltiazem},\ent{cbb}\})
\mid  (\{\ents{get}{verapamil}\},\{\ents{stop}{cbb}\},\{\ent{verapamil},\ent{cbb}\})
\\[-4pt] & \mid &  (\{\ent{verapamil}\},\{\ents{stop}{cbb}\},\{\ent{verapamil},\ent{cbb}\})
\mid  (\{\ent{diltiazem},\ent{verapamil}\},\{\ents{stop}{cbb}\},\{\ents{alert}{dup}\})
%
\mid  (\{\ents{get}{propranolol}\},\{\ents{stop}{nsbb}\},\{\ent{propranolol},\ent{nsbb}\})
\\[-4pt] & \mid &  (\{\ent{propranolol}\},\{\ents{stop}{nsbb}\},\{\ent{propranolol},\ent{nsbb}\})
\mid  (\{\ents{get}{carvedilol}\},\{\ents{stop}{nsbb}\},\{\ent{carvedilol},\ent{nsbb}\})
\mid  (\{\ent{carvedilol}\},\{\ents{stop}{nsbb}\},\{\ent{carvedilol},\ent{nsbb}\})
\\[-4pt] & \mid &  (\{\ent{propranolol},\ent{carvedilol}\},\{\ents{stop}{nsbb}\},\{\ents{alert}{dup}\})
%
\mid  (\{\ents{get}{bisoprolol}\},\{\ents{stop}{sbb}\},\{\ent{bisoprolol},\ent{sbb}\})
\mid  (\{\ent{bisoprolol}\},\{\ents{stop}{sbb}\},\{\ent{bisoprolol},\ent{sbb}\})
\\[-4pt] & \mid &  (\{\ents{get}{atenolol}\},\{\ents{stop}{sbb}\},\{\ent{atenolol},\ent{sbb}\})
\mid  (\{\ent{atenolol}\},\{\ents{stop}{sbb}\},\{\ent{atenolol},\ent{sbb}\})
\mid  (\{\ent{bisoprolol},\ent{atenolol}\},\{\ents{stop}{sbb}\},\{\ents{alert}{dup}\})
%
\\[-4pt] & \mid &  (\{\ents{get}{flecainide}\},\{\ents{stop}{flec}\},\{\ent{flecainide}\})
\mid  (\{\ent{flecainide}\},\{\ents{stop}{flec}\},\{\ent{flecainide}\})
\mid  (\{\ents{get}{warfarin}\},\{\ents{stop}{warf}\},\{\ent{warfarin}\})
\\[-4pt] & \mid &  (\{\ent{warfarin}\},\{\ents{stop}{warf}\},\{\ent{warfarin}\})
%
\mid  (\{\ents{get}{apixaban}\},\{\ents{stop}{doac}\},\{\ent{apixaban},\ent{doac}\})
\mid  (\{\ent{apixaban}\},\{\ents{stop}{doac}\},\{\ent{apixaban},\ent{doac}\})
\\[-4pt] & \mid &  (\{\ents{get}{dabigatran}\},\{\ents{stop}{doac}\},\{\ent{dabigatran},\ent{doac}\})
\mid  (\{\ent{dabigatran}\},\{\ents{stop}{doac}\},\{\ent{dabigatran},\ent{doac}\})
\mid  (\{\ent{apixaban},\ent{dabigatran}\},\{\ents{stop}{doac}\},\{\ents{alert}{dup}\})
%
\\[-4pt] & \mid &  (\{\ents{get}{vkant}\},\{\ents{stop}{vkant}\},\{\ent{vkant}\})
\mid  (\{\ent{vkant}\},\{\ents{stop}{vkant}\},\{\ent{vkant}\})
%
\mid  (\{\ents{get}{benazepril}\},\{\ents{stop}{acei}\},\{\ent{benazepril},\ent{acei}\})
\\[-4pt] & \mid &  (\{\ent{benazepril}\},\{\ents{stop}{acei}\},\{\ent{benazepril},\ent{acei}\})
\mid  (\{\ents{get}{captopril}\},\{\ents{stop}{acei}\},\{\ent{captopril},\ent{acei}\})
\mid  (\{\ent{captopril}\},\{\ents{stop}{acei}\},\{\ent{captopril},\ent{acei}\})
\\[-4pt] & \mid &  (\{\ent{benazepril},\ent{captopril}\},\{\ents{stop}{acei}\},\{\ents{alert}{dup}\})
%
\mid  (\{\ents{get}{olmesortan}\},\{\ents{stop}{arb}\},\{\ent{olmesortan},\ent{arb}\})
\mid  (\{\ent{olmesortan}\},\{\ents{stop}{arb}\},\{\ent{olmesortan},\ent{arb}\})
\\[-4pt] & \mid &  (\{\ents{get}{irbesartan}\},\{\ents{stop}{arb}\},\{\ent{irbesartan},\ent{arb}\})
\mid  (\{\ent{irbesartan}\},\{\ents{stop}{arb}\},\{\ent{irbesartan},\ent{arb}\})
\mid  (\{\ent{olmesortan},\ent{irbesartan}\},\{\ents{stop}{arb}\},\{\ents{alert}{dup}\})
%
\\[-4pt] & \mid &  (\{\ents{get}{indapamide}\},\{\ents{stop}{td}\},\{\ent{indapamide},\ent{td}\})
\mid  (\{\ent{indapamide}\},\{\ents{stop}{td}\},\{\ent{indapamide},\ent{td}\})
\mid  (\{\ents{get}{chlorothiazide}\},\{\ents{stop}{td}\},\{\ent{chlorothiazide},\ent{td}\})
\\[-4pt] & \mid &  (\{\ent{chlorothiazide}\},\{\ents{stop}{td}\},\{\ent{chlorothiazide},\ent{td}\})
\mid  (\{\ent{indapamide},\ent{chlorothiazide}\},\{\ents{stop}{td}\},\{\ents{alert}{dup}\})
%
\mid  (\{\ent{doac}\},\{\ents{doac}{ok},\ents{doac}{fail}\},\{\ents{doac}{test}\})
\\[-4pt] & \mid &  (\{\ents{doac}{ok}\},\{\ents{doac}{fail}\},\{\ents{doac}{ok}\})
\mid  (\{\ents{doac}{fail}\},\{\ents{doac}{ok}\},\{\ents{doac}{fail}\})
\mid  (\{\ent{doac}\},\{\ents{doac}{fail},\ents{stop}{doac}\},\{\ents{doac}{danger}\})
\\[-4pt] & \mid &  (\{\ent{doac}\},\{\ents{doac}{danger},\ents{stop}{doac}\},\{\ent{danger}\})
\\[-4pt]
\ent{ADR} &  \triangleq 
& (\{\ents{get}{apixaban},\ents{get}{diltiazem}\},\varnothing,\{\ent{moderate}\})
\mid  (\{\ents{get}{apixaban},\ent{diltiazem}\},\varnothing,\{\ent{moderate}\})
\mid  (\{\ent{apixaban},\ents{get}{diltiazem}\},\varnothing,\{\ent{moderate}\})
\\[-4pt] & \mid &  (\{\ent{apixaban},\ent{diltiazem}\},\varnothing,\{\ent{moderate}\})
\mid  (\{\ents{get}{apixaban},\ents{get}{verapamil}\},\varnothing,\{\ent{moderate}\})
\mid  (\{\ents{get}{apixaban},\ent{verapamil}\},\varnothing,\{\ent{moderate}\})
\\[-4pt] & \mid &  (\{\ent{apixaban},\ents{get}{verapamil}\},\varnothing,\{\ent{moderate}\})
\mid  (\{\ent{apixaban},\ent{verapamil}\},\varnothing,\{\ent{moderate}\})
\mid  (\{\ents{get}{dabigatran},\ents{get}{diltiazem}\},\varnothing,\{\ent{moderate}\})
\\[-4pt] & \mid &  (\{\ents{get}{dabigatran},\ent{diltiazem}\},\varnothing,\{\ent{moderate}\})
\mid  (\{\ent{dabigatran},\ents{get}{diltiazem}\},\varnothing,\{\ent{moderate}\})
\mid  (\{\ent{dabigatran},\ent{diltiazem}\},\varnothing,\{\ent{moderate}\})
\\[-4pt] & \mid &  (\{\ents{get}{dabigatran},\ents{get}{verapamil}\},\varnothing,\{\ent{major}\})
\mid  (\{\ents{get}{dabigatran},\ent{verapamil}\},\varnothing,\{\ent{major}\})
\mid  (\{\ent{dabigatran},\ents{get}{verapamil}\},\varnothing,\{\ent{major}\})
\\[-4pt] & \mid &  (\{\ent{dabigatran},\ent{verapamil}\},\varnothing,\{\ent{major}\})
\mid  (\{\ents{get}{dabigatran},\ents{get}{carvedilol}\},\varnothing,\{\ent{moderate}\})
\mid  (\{\ents{get}{dabigatran},\ent{carvedilol}\},\varnothing,\{\ent{moderate}\})
\\[-4pt] & \mid &  (\{\ent{dabigatran},\ents{get}{carvedilol}\},\varnothing,\{\ent{moderate}\})
\mid  (\{\ent{dabigatran},\ent{carvedilol}\},\varnothing,\{\ent{moderate}\})
\mid  (\{\ents{get}{warfarin},\ents{get}{benazepril}\},\varnothing,\{\ent{minor}\})
\\[-4pt] & \mid &  (\{\ents{get}{warfarin},\ent{benazepril}\},\varnothing,\{\ent{minor}\})
\mid  (\{\ent{warfarin},\ents{get}{benazepril}\},\varnothing,\{\ent{minor}\})
\mid  (\{\ent{warfarin},\ent{benazepril}\},\varnothing,\{\ent{minor}\})
\\[-4pt] & \mid &  (\{\ents{get}{warfarin},\ents{get}{indapamide}\},\varnothing,\{\ent{minor}\})
\mid  (\{\ents{get}{warfarin},\ent{indapamide}\},\varnothing,\{\ent{minor}\})
\mid  (\{\ent{warfarin},\ents{get}{indapamide}\},\varnothing,\{\ent{minor}\})
\\[-4pt] & \mid &  (\{\ent{warfarin},\ent{indapamide}\},\varnothing,\{\ent{minor}\})
\mid  (\{\ents{get}{warfarin},\ents{get}{chlorothiazide}\},\varnothing,\{\ent{minor}\})
\mid  (\{\ents{get}{warfarin},\ent{chlorothiazide}\},\varnothing,\{\ent{minor}\})
\\[-4pt] & \mid &  (\{\ent{warfarin},\ents{get}{chlorothiazide}\},\varnothing,\{\ent{minor}\})
\mid  (\{\ent{warfarin},\ent{chlorothiazide}\},\varnothing,\{\ent{minor}\})
\mid  (\{\ents{get}{warfarin},\ents{get}{propranolol}\},\varnothing,\{\ent{minor}\})
\\[-4pt] & \mid &  (\{\ents{get}{warfarin},\ent{propranolol}\},\varnothing,\{\ent{minor}\})
\mid  (\{\ent{warfarin},\ents{get}{propranolol}\},\varnothing,\{\ent{minor}\})
\mid  (\{\ent{warfarin},\ent{propranolol}\},\varnothing,\{\ent{minor}\})
\\[-4pt] & \mid &  (\{\ents{get}{flecainide},\ents{get}{diltiazem}\},\varnothing,\{\ent{major}\})
\mid  (\{\ents{get}{flecainide},\ent{diltiazem}\},\varnothing,\{\ent{major}\})
\mid  (\{\ent{flecainide},\ents{get}{diltiazem}\},\varnothing,\{\ent{major}\})
\\[-4pt] & \mid &  (\{\ent{flecainide},\ent{diltiazem}\},\varnothing,\{\ent{major}\})
\mid  (\{\ents{get}{flecainide},\ents{get}{verapamil}\},\varnothing,\{\ent{major}\})
\mid  (\{\ents{get}{flecainide},\ent{verapamil}\},\varnothing,\{\ent{major}\})
\\[-4pt] & \mid &  (\{\ent{flecainide},\ents{get}{verapamil}\},\varnothing,\{\ent{major}\})
\mid  (\{\ent{flecainide},\ent{verapamil}\},\varnothing,\{\ent{major}\})
\mid  (\{\ents{get}{flecainide},\ents{get}{bisoprolol}\},\varnothing,\{\ent{moderate}\})
\\[-4pt] & \mid &  (\{\ents{get}{flecainide},\ent{bisoprolol}\},\varnothing,\{\ent{moderate}\})
\mid  (\{\ent{flecainide},\ents{get}{bisoprolol}\},\varnothing,\{\ent{moderate}\})
\mid  (\{\ent{flecainide},\ent{bisoprolol}\},\varnothing,\{\ent{moderate}\})
\\[-4pt] & \mid &  (\{\ents{get}{flecainide},\ents{get}{atenolol}\},\varnothing,\{\ent{moderate}\})
\mid  (\{\ents{get}{flecainide},\ent{atenolol}\},\varnothing,\{\ent{moderate}\})
\mid  (\{\ent{flecainide},\ents{get}{atenolol}\},\varnothing,\{\ent{moderate}\})
\\[-4pt] & \mid &  (\{\ent{flecainide},\ent{atenolol}\},\varnothing,\{\ent{moderate}\})
\mid  (\{\ents{get}{flecainide},\ents{get}{propranolol}\},\varnothing,\{\ent{moderate}\})
\mid  (\{\ents{get}{flecainide},\ent{propranolol}\},\varnothing,\{\ent{moderate}\})
\\[-4pt] & \mid &  (\{\ent{flecainide},\ents{get}{propranolol}\},\varnothing,\{\ent{moderate}\})
\mid  (\{\ent{flecainide},\ent{propranolol}\},\varnothing,\{\ent{moderate}\})
\mid  (\{\ents{get}{flecainide},\ents{get}{carvedilol}\},\varnothing,\{\ent{moderate}\})
\\[-4pt] & \mid &  (\{\ents{get}{flecainide},\ent{carvedilol}\},\varnothing,\{\ent{moderate}\})
\mid  (\{\ent{flecainide},\ents{get}{carvedilol}\},\varnothing,\{\ent{moderate}\})
\mid  (\{\ent{flecainide},\ent{carvedilol}\},\varnothing,\{\ent{moderate}\})
\\[-4pt] & \mid &  (\{\ent{major}\},\varnothing,\{\ent{major}\})
\mid  (\{\ent{moderate}\},\varnothing,\{\ent{moderate}\})
\mid  (\{\ent{minor}\},\varnothing,\{\ent{minor}\})
\mid  (\{\ents{alert}{dup}\},\varnothing,\{\ents{alert}{dup}\})
\mid  (\{\ent{danger}\},\varnothing,\{\ent{danger}\})
\end{array}
\]
\normalsize
\caption{Reactions for the comorbidity case study in~\Cref{sec:cmsb2024}.}
\label{fig:bioresolve:comorbidities:reactions}
\end{figure*}

\begin{figure*}[t]
\fontsize{6}{0}
\[
\begin{array}{rcl}
\ent{eafib1} &  \triangleq 
& (\varnothing,\{\ent{afib}\},\varnothing).\ent{eafib1} 
+ (\{\ent{afib}\},\varnothing,\varnothing).\ent{ehr}
\\[-4pt]
\ent{ehr} &  \triangleq 
& (\varnothing,\{\ents{heart}{rate}\},\varnothing).\ent{ehr} 
+ (\{\ents{heart}{rate}\},\varnothing,\varnothing).\ent{ebb}
\\[-4pt]
\ent{ebb} &  \triangleq 
& \varnothing.\ent{ebb} + \ents{e}{cbb} + \ents{e}{nsbb} + \ents{e}{sbb}
\\[-4pt]
\ents{e}{cbb} &  \triangleq 
& (\varnothing,\{\ent{verapamil}\},\{\ents{get}{diltiazem}\}).\ent{empty} 
+ (\varnothing,\{\ent{diltiazem}\},\{\ents{get}{verapamil}\}).\ent{empty}
\\[-4pt]
\ents{e}{nsbb} &  \triangleq 
& (\varnothing,\{\ent{carvedilol}\},\{\ents{get}{propranolol}\}).\ent{empty} 
+ (\varnothing,\{\ent{propranolol}\},\{\ents{get}{carvedilol}\}).\ent{empty}
\\[-4pt]
\ents{e}{sbb} &  \triangleq 
& (\varnothing,\{\ent{atenolol}\},\{\ents{get}{bisoprolol}\}).\ent{empty} 
+ (\varnothing,\{\ent{bisoprolol}\},\{\ents{get}{atenolol}\}).\ent{empty}
\\[-4pt]
\ent{eafib2} &  \triangleq 
& (\varnothing,\{\ent{afib}\},\varnothing).\ent{eafib2} 
+ (\{\ent{afib}\},\varnothing,\varnothing).\ent{ehf}
\\[-4pt]
\ent{ehf} &  \triangleq 
& (\varnothing,\{\ents{has}{fib}\},\varnothing).\ent{ehf} 
+ (\{\ents{has}{fib}\},\varnothing,\varnothing).\ent{eflec}
\\[-4pt]
\ent{eflec} &  \triangleq 
& \varnothing.\ent{eflec} + \ents{e}{flec}
\\[-4pt]
\ents{e}{flec} &  \triangleq 
& \{\ents{get}{flecainide}\}.\ent{empty}
\\[-4pt]
\ent{eafib3} &  \triangleq 
& (\varnothing,\{\ent{afib}\},\varnothing).\ent{eafib3} 
+ (\{\ent{afib}\},\varnothing,\varnothing).\ent{econs}
\\[-4pt]
\ent{econs} &  \triangleq 
& (\varnothing,\{\ents{heart}\ent{rate},\ents{has}{fib}\},\varnothing).\ent{econs} 
+ (\varnothing,\{\ents{consensus}{acei}\},\varnothing).\ent{econs} 
+ (\{\ents{consensus}{acei},\ents{heart}{rate}\},\varnothing,\varnothing).\ent{estroke} 
+ (\{\ents{consensus}{acei},\ents{has}{fib}\},\varnothing,\varnothing).\ent{estroke} 
\\[-4pt]
\ent{estroke} &  \triangleq 
& (\varnothing,\{\ent{diseases},\ent{over75}\},\varnothing).\ent{ewarf} 
+ (\{\ent{over75}\},\{\ents{doac}{fail},\ents{doac}{int}\},\varnothing).\ent{edoac} 
+ (\{\ent{diseases}\},\{\ents{doac}{fail},\ents{doac}{int}\},\varnothing).\ent{edoac} 
+ (\{\ent{over75},\ents{doac}{fail}\},\varnothing,\varnothing).\ent{evkant} 
\\[-4pt] 
& + & (\{\ent{over75},\ents{doac}{int}\},\varnothing,\varnothing).\ent{evkant} 
+ (\{\ent{diseases}\},\{\ents{doac}{fail},\varnothing,\varnothing).\ent{evkant} 
+ (\{\ent{diseases}\},\{\ents{doac}{int},\varnothing,\varnothing).\ent{evkant}
\\[-4pt]
\ent{ewarf} &  \triangleq 
& \varnothing.\ent{ewarf} + \ents{e}{warf}
\\[-4pt]
\ents{e}{warf} &  \triangleq 
& \{\ents{get}{warfarin}\}.\ent{empty}
\\[-4pt]
\ent{edoac} &  \triangleq 
& \varnothing.\ent{edoac} + \ents{e}{doac}
\\[-4pt]
\ents{e}{doac} &  \triangleq 
& (\varnothing,\{\ent{dabigatran}\},\{\ents{get}{apixaban}\}).\ents{e}{doacfail} 
+ (\varnothing,\{\ent{apixaban}\},\{\ents{get}{dabigatran}\}).\ents{e}{doacfail}
\\[-4pt]
\ents{e}{doacfail} &  \triangleq 
& (\{\ents{doac}{fail}\},\varnothing,\{\ents{stop}{doac}\}).\ent{evkant} 
+ (\varnothing,\{\ents{doac}{fail}\},\varnothing).\ents{e}{doacfail}
\\[-4pt]
\ent{evkant} &  \triangleq 
& \varnothing.\ent{evkant} + \ents{e}{vkant}
\\[-4pt]
\ents{e}{vkant} &  \triangleq 
& \{\ents{get}{vkant}\}.\ent{empty}
\\[-4pt]
\ent{ghyper} &  \triangleq 
& (\varnothing,\{\ent{hyper}\},\varnothing).\ent{ghyper} 
+ (\{\ent{hyper}\},\varnothing,\varnothing).\ent{g1}
\\[-4pt]
\ent{g1} &  \triangleq 
& (\{\ent{diabete}\},\varnothing,\varnothing).\ent{g2} 
+ (\{\ent{below55}\},\{\ent{diabete},\ent{origin}\},\varnothing).\ent{g2} 
+ (\varnothing,\{\ent{below55},\ent{diabete}\},\varnothing).\ent{g3} 
+ (\{\ent{origin}\},\{\ent{diabete}\},\varnothing).\ent{g3}
\\[-4pt]
\ent{g2} &  \triangleq 
& \varnothing.\ent{g2} 
+ (\varnothing,\{\ent{captopril}\},\{\ents{get}{benazepril}\}).\ent{g4}
+ (\varnothing,\{\ent{benazepril}\},\{\ents{get}{captopril}\}).\ent{g4}
+ (\varnothing,\{\ent{irbesartan}\},\{\ents{get}{olmesortan}\}).\ent{g5}
\\[-4pt]
& + & (\varnothing,\{\ent{olmesortan}\},\{\ents{get}{irbesartan}\}).\ent{g5}
\\[-4pt]
\ent{g3} &  \triangleq 
& \varnothing.\ent{g3} 
+ (\varnothing,\{\ent{verapamil}\},\{\ents{get}{diltiazem}\}).\ent{g6}
+ (\varnothing,\{\ent{diltiazem}\},\{\ents{get}{verapamil}\}).\ent{g6}
\\[-4pt]
\ent{g4} &  \triangleq 
& \varnothing.\ent{g4} 
+ (\varnothing,\{\ent{verapamil}\},\{\ents{get}{diltiazem}\}).\ent{g7}
+ (\varnothing,\{\ent{diltiazem}\},\{\ents{get}{verapamil}\}).\ent{g7}
+ (\varnothing,\{\ent{chlorothiazide}\},\{\ents{get}{indapamide}\}).\ent{g8}
\\[-4pt]
& + & (\varnothing,\{\ent{indapamide}\},\{\ents{get}{chlorothiazide}\}).\ent{g8}
\\[-4pt]
\ent{g5} &  \triangleq 
& \varnothing.\ent{g5} 
+ (\varnothing,\{\ent{verapamil}\},\{\ents{get}{diltiazem}\}).\ent{g9}
+ (\varnothing,\{\ent{diltiazem}\},\{\ents{get}{verapamil}\}).\ent{g9}
+ (\varnothing,\{\ent{chlorothiazide}\},\{\ents{get}{indapamide}\}).\ent{g10}
\\[-4pt]
& + & (\varnothing,\{\ent{indapamide}\},\{\ents{get}{chlorothiazide}\}).\ent{g10}
\\[-4pt]
\ent{g6} &  \triangleq 
& \varnothing.\ent{g6} 
+ (\varnothing,\{\ent{captopril}\},\{\ents{get}{benazepril}\}).\ent{g7}
+ (\varnothing,\{\ent{benazepril}\},\{\ents{get}{captopril}\}).\ent{g7}
+ (\varnothing,\{\ent{irbesartan}\},\{\ents{get}{olmesortan}\}).\ent{g9}
\\[-4pt]
& + & (\varnothing,\{\ent{olmesortan}\},\{\ents{get}{irbesartan}\}).\ent{g9}
+ (\varnothing,\{\ent{chlorothiazide}\},\{\ents{get}{indapamide}\}).\ent{g11}
+ (\varnothing,\{\ent{indapamide}\},\{\ents{get}{chlorothiazide}\}).\ent{g11}
\\[-4pt]
\ent{g7} &  \triangleq 
& \varnothing.\ent{g7} 
+ (\varnothing,\{\ent{irbesartan}\},\{\ents{get}{olmesortan}\}).\ent{etd}
+ (\varnothing,\{\ent{olmesortan}\},\{\ents{get}{irbesartan}\}).\ent{etd}
+ (\varnothing,\{\ent{chlorothiazide}\},\{\ents{get}{indapamide}\}).\ent{earb}
\\[-4pt]
& + & (\varnothing,\{\ent{indapamide}\},\{\ents{get}{chlorothiazide}\}).\ent{earb}
\\[-4pt]
\ent{g8} &  \triangleq 
& \varnothing.\ent{g8} 
+ (\varnothing,\{\ent{irbesartan}\},\{\ents{get}{olmesortan}\}).\ent{ecbb}
+ (\varnothing,\{\ent{olmesortan}\},\{\ents{get}{irbesartan}\}).\ent{ecbb}
+ (\varnothing,\{\ent{verapamil}\},\{\ents{get}{diltiazem}\}).\ent{earb}
\\[-4pt]
& + & (\varnothing,\{\ent{diltiazem}\},\{\ents{get}{verapamil}\}).\ent{earb}
\\[-4pt]
\ent{g9} &  \triangleq 
& \varnothing.\ent{g9} 
+ (\varnothing,\{\ent{captopril}\},\{\ents{get}{benazepril}\}).\ent{etd}
+ (\varnothing,\{\ent{benazepril}\},\{\ents{get}{captopril}\}).\ent{etd}
+ (\varnothing,\{\ent{chlorothiazide}\},\{\ents{get}{indapamide}\}).\ent{eacei}
\\[-4pt]
& + & (\varnothing,\{\ent{indapamide}\},\{\ents{get}{chlorothiazide}\}).\ent{eacei}
\\[-4pt]
\ent{g10} &  \triangleq 
& \varnothing.\ent{g10} 
+ (\varnothing,\{\ent{captopril}\},\{\ents{get}{benazepril}\}).\ent{ecbb}
+ (\varnothing,\{\ent{benazepril}\},\{\ents{get}{captopril}\}).\ent{ecbb}
+ (\varnothing,\{\ent{chlorothiazide}\},\{\ents{get}{indapamide}\}).\ent{eacei}
\\[-4pt]
& + & (\varnothing,\{\ent{indapamide}\},\{\ents{get}{chlorothiazide}\}).\ent{eacei}
\\[-4pt]
\ent{g11} &  \triangleq 
& \varnothing.\ent{g11} 
+ (\varnothing,\{\ent{captopril}\},\{\ents{get}{benazepril}\}).\ent{earb}
+ (\varnothing,\{\ent{benazepril}\},\{\ents{get}{captopril}\}).\ent{earb}
+ (\varnothing,\{\ent{irbesartan}\},\{\ents{get}{olmesortan}\}).\ent{eacei}
\\[-4pt]
& + & (\varnothing,\{\ent{olmesortan}\},\{\ents{get}{irbesartan}\}).\ent{eacei}
\\[-4pt]
\ent{ecbb} &  \triangleq 
& \varnothing.\ent{ecbb} + \ents{e}{cbb}
\\[-4pt]
\ent{eacei} &  \triangleq 
& \varnothing.\ent{eacei} + \ents{e}{acei}
\\[-4pt]
\ents{e}{acei} &  \triangleq 
& (\varnothing,\{\ent{captopril}\},\{\ents{get}{benazepril}\}).\ent{empty} 
+ (\varnothing,\{\ent{benazepril}\},\{\ents{get}{captopril}\}).\ent{empty}
\\[-4pt]
\ent{earb} &  \triangleq 
& \varnothing.\ent{earb} + \ents{e}{arb}
\\[-4pt]
\ents{e}{arb} &  \triangleq 
& (\varnothing,\{\ent{irbesartan}\},\{\ents{get}{olmesortan}\}).\ent{empty} 
+ (\varnothing,\{\ent{olmesortan}\},\{\ents{get}{irbesartan}\}).\ent{empty}
\\[-4pt]
\ent{etd} &  \triangleq 
& \varnothing.\ent{etd} + \ents{e}{td}
\\[-4pt]
\ents{e}{td} &  \triangleq 
& (\varnothing,\{\ent{chlorothiazide}\},\{\ents{get}{indapamide}\}).\ent{empty} 
+ (\varnothing,\{\ent{indapamide}\},\{\ents{get}{chlorothiazide}\}).\ent{empty}
\\[-4pt]
\ents{k}{doac} &  \triangleq 
& (\{\ents{doac}{test}\},\varnothing,\{\ents{doac}{ok}\}).\ent{empty} 
+ (\{\ents{doac}{test}\},\varnothing,\{\ents{doac}{fail}\}).\ent{empty}
+ (\varnothing,\{\ents{doac}{test}\},\varnothing).\ents{k}{doac}
\\[-4pt]
\ent{empty} &  \triangleq 
& \varnothing.\ent{empty}
\\[-4pt]
\ent{kafib} &  \triangleq 
& \{\ent{afib}\}.\ent{empty} + \ent{empty}
\\[-4pt]
\ent{khf} &  \triangleq 
& \{\ents{has}{fib}\}.\ent{empty} + \ent{empty}
\\[-4pt]
\ent{khr} &  \triangleq 
& \{\ents{heart}{rate}\}.\ent{empty} + \ent{empty}
\\[-4pt]
\ent{kcons} &  \triangleq 
& \{\ents{consensus}{acei}\}.\ent{empty} + \ent{empty}
\\[-4pt]
\ent{kage} &  \triangleq 
& \{\ent{over75}\}.\ent{empty} + \{\ent{below55}\}.\ent{empty} + \ent{empty}
\\[-4pt]
\ent{kdiabete} &  \triangleq 
& \{\ent{diabete}\}.\ent{empty} + \ent{empty}
\\[-4pt]
\ent{kdoacint} &  \triangleq 
& \{\ents{doac}{int}\}.\ent{empty} + \ent{empty}
\\[-4pt]
\ent{khyper} &  \triangleq 
& \{\ent{hyper}\}.\ent{empty} + \ent{empty}
\\[-4pt]
\ent{korigin} &  \triangleq 
& \{\ent{origin}\}.\ent{empty} + \ent{empty}
\end{array}
\]
\normalsize
\caption{Context process definitions for the comorbidity case study in~\Cref{sec:cmsb2024}.
The initial context is given by the parallel composition of therapies 
$\ent{eafib1}
\mid \ent{eafib2}
\mid \ent{eafib3}
\mid \ent{ghyper}$ and the parallel composition of features
$\ent{kafib}
\mid \ent{khf}
\mid \ent{khr}
\mid \ent{kcons}
\mid \ent{kage}
\mid \ent{kdiabete}
\mid \ent{kdoacint}
\mid \ent{khyper}
\mid \ent{korigin}
\mid \ents{k}{doac}$.}
\label{fig:bioresolve:comorbidities:contexts}
\end{figure*}
%\begin{verbatim}
%myentities([]).
%
%myreactions([
%  react([hyper],[],[hyper]), react([afib],[],[afib]), react([has_fib],[],[has_fib]), react([heart_rate],[],[heart_rate]),
%  react([consensus_acei],[],[consensus_acei]), react([over75],[],[over75]), react([below55],[],[below55]), react([diabete],[],[diabete]),
%  react([origin],[],[origin]), react([doac_int],[],[doac_int]), react([doac],[doac_ok,doac_fail],[doac_test]), react([doac_ok],[doac_fail],[doac_ok]),
%  react([doac_fail],[doac_ok],[doac_fail]), react([hyper],[],[diseases]), react([diabete],[],[diseases]), react([get_diltiazem],[stop_cbb],[diltiazem,cbb]),
%  react([diltiazem],[stop_cbb],[diltiazem,cbb]), react([get_verapamil],[stop_cbb],[verapamil,cbb]), react([verapamil],[stop_cbb],[verapamil,cbb]),
%  react([diltiazem,verapamil],[stop_cbb],[alert_dup]), react([get_propranolol],[stop_nsbb],[propranolol,nsbb]),
%  react([propranolol],[stop_nsbb],[propranolol,nsbb]), react([get_carvedilol],[stop_nsbb],[carvedilol,nsbb]), react([carvedilol],[stop_nsbb],[carvedilol,nsbb]), 
%  react([propranolol,carvedilol],[stop_nsbb],[alert_dup]), react([get_bisoprolol],[stop_sbb],[bisoprolol,sbb]), react([bisoprolol],[stop_sbb],[bisoprolol,sbb]),
%  react([get_atenolol],[stop_sbb],[atenolol,sbb]), react([atenolol],[stop_sbb],[atenolol,sbb]), react([bisoprolol,atenolol],[stop_sbb],[alert_dup]),
%  react([get_flecainide],[stop_flec],[flecainide]), react([flecainide],[stop_flec],[flecainide]), react([get_warfarin],[stop_warf],[warfarin]),
%  react([warfarin],[stop_warf],[warfarin]), react([get_apixaban],[stop_doac],[apixaban,doac]), react([apixaban],[stop_doac],[apixaban,doac]),
%  react([get_dabigatran],[stop_doac],[dabigatran,doac]), react([dabigatran],[stop_doac],[dabigatran,doac]), react([apixaban,dabigatran],[stop_doac],[alert_dup]),
%  react([get_vkant],[stop_vkant],[vkant]), react([vkant],[stop_vkant],[vkant]), react([get_benazepril],[stop_acei],[benazepril,acei]), 
%  react([benazepril],[stop_acei],[benazepril,acei]), react([get_captopril],[stop_acei],[captopril,acei]), react([captopril],[stop_acei],[captopril,acei]),
%  react([benazepril,captopril],[stop_acei],[alert_dup]), react([get_olmesortan],[stop_arb],[olmesortan,arb]), react([olmesortan],[stop_arb],[olmesortan,arb]), 
%  react([get_irbesartan],[stop_arb],[irbesartan,arb]), react([irbesartan],[stop_arb],[irbesartan,arb]), react([olmesortan,irbesartan],[stop_arb],[alert_dup]),
%  react([get_indapamide],[stop_td],[indapamide,td]), react([indapamide],[stop_td],[indapamide,td]), react([get_chlorothiazide],[stop_td],[chlorothiazide,td]), 
%  react([chlorothiazide],[stop_td],[chlorothiazide,td]), react([indapamide,chlorothiazide],[stop_td],[alert_dup]), react([doac,doac_fail],[stop_doac],[doac_danger]),
%  react([doac,doac_danger],[stop_doac],[danger]), react([get_apixaban,get_diltiazem],[],[moderate]), react([get_apixaban,diltiazem],[],[moderate]),
%  react([apixaban,get_diltiazem],[],[moderate]), react([apixaban,diltiazem],[],[moderate]), react([get_apixaban,get_verapamil],[],[moderate]),
%  react([get_apixaban,verapamil],[],[moderate]), react([apixaban,get_verapamil],[],[moderate]), react([apixaban,verapamil],[],[moderate]),
%  react([get_dabigatran,get_diltiazem],[],[moderate]), react([get_dabigatran,diltiazem],[],[moderate]), react([dabigatran,get_diltiazem],[],[moderate]),
%  react([dabigatran,diltiazem],[],[moderate]), react([get_dabigatran,get_verapamil],[],[major]), react([get_dabigatran,verapamil],[],[major]),
%  react([dabigatran,get_verapamil],[],[major]), react([dabigatran,verapamil],[],[major]), react([get_dabigatran,get_carvedilol],[],[moderate]),
%  react([get_dabigatran,carvedilol],[],[moderate]), react([dabigatran,get_carvedilol],[],[moderate]), react([dabigatran,carvedilol],[],[moderate]), 
%  react([get_warfarin,get_benazepril],[],[minor]), react([get_warfarin,benazepril],[],[minor]), react([warfarin,get_benazepril],[],[minor]),
%  react([warfarin,benazepril],[],[minor]), react([get_warfarin,get_indapamide],[],[minor]), react([get_warfarin,indapamide],[],[minor]),
%  react([warfarin,get_indapamide],[],[minor]), react([warfarin,indapamide],[],[minor]), react([get_warfarin,get_chlorothiazide],[],[minor]),
%  react([get_warfarin,chlorothiazide],[],[minor]), react([warfarin,get_chlorothiazide],[],[minor]), react([warfarin,chlorothiazide],[],[minor]),
%  react([get_warfarin,get_propranolol],[],[minor]), react([get_warfarin,propranolol],[],[minor]), react([warfarin,get_propranolol],[],[minor]),
%  react([warfarin,propranolol],[],[minor]), react([get_flecainide,get_diltiazem],[],[major]), react([get_flecainide,diltiazem],[],[major]),
%  react([flecainide,get_diltiazem],[],[major]), react([flecainide,diltiazem],[],[major]), react([get_flecainide,get_verapamil],[],[major]),
%  react([get_flecainide,verapamil],[],[major]), react([flecainide,get_verapamil],[],[major]), react([flecainide,verapamil],[],[major]), 
%  react([get_flecainide,get_bisoprolol],[],[moderate]), react([get_flecainide,bisoprolol],[],[moderate]), react([flecainide,get_bisoprolol],[],[moderate]),
%  react([flecainide,bisoprolol],[],[moderate]), react([get_flecainide,get_atenolol],[],[moderate]), react([get_flecainide,atenolol],[],[moderate]), 
%  react([flecainide,get_atenolol],[],[moderate]), react([flecainide,atenolol],[],[moderate]), react([get_flecainide,get_propranolol],[],[moderate]),
%  react([get_flecainide,propranolol],[],[moderate]), react([flecainide,get_propranolol],[],[moderate]), react([flecainide,propranolol],[],[moderate]), 
%  react([get_flecainide,get_carvedilol],[],[moderate]), react([get_flecainide,carvedilol],[],[moderate]), react([flecainide,get_carvedilol],[],[moderate]),
%  react([flecainide,carvedilol],[],[moderate]), react([major],[],[major]), react([moderate],[],[moderate]),
%  react([minor],[],[minor]), react([alert_dup],[],[alert_dup]), react([danger],[],[danger]) ]).
%
%mycontext("[eafib1,eafib2,eafib3,ghyper,kafib,khf,khr,kcons,kage,kdiabete,kdoacint,khyper,korigin,k_doac]").
%
%myenvironment("[
%  eafib1 = (?{},{afib},{}?.eafib1 + ?{afib},{},{}?.ehr),
%  ehr = (?{},{heart_rate},{}?.ehr + ?{heart_rate},{},{}?.ebb),
%  ebb = ({}.ebb + e_cbb + e_nsbb + e_sbb),
%  e_cbb = (?{},{verapamil},{get_diltiazem}?.empty + ?{},{diltiazem},{get_verapamil}?.empty),
%  e_nsbb = (?{},{carvedilol},{get_propranolol}?.empty + ?{},{propranolol},{get_carvedilol}?.empty),
%  e_sbb = (?{},{atenolol},{get_bisoprolol}?.empty + ?{},{bisoprolol},{get_atenolol}?.empty),
%  eafib2 = (?{},{afib},{}?.eafib2 + ?{afib},{},{}?.ehf),
%  ehf = (?{},{has_fib},{}?.ehf + ?{has_fib},{},{}?.eflec),
%  eflec = ({}.eflec + e_flec),
%  e_flec = {get_flecainide}.empty,
%  eafib3 = (?{},{afib},{}?.eafib3 + ?{afib},{},{}?.econs),
%  econs = (?{},{heart_rate,has_fib},{}?.econs + ?{},{consensus_acei},{}?.econs 
%           + ?{consensus_acei,heart_rate},{},{}?.estroke + ?{consensus_acei,has_fib},{},{}?.estroke),
%  estroke = (?{},{diseases,over75},{}?.ewarf + ?{over75},{doac_fail,doac_int},{}?.edoac + ?{diseases},{doac_fail,doac_int},{}?.edoac 
%           + ?{over75,doac_fail},{},{}?.evkant + ?{over75,doac_int},{},{}?.evkant + ?{diseases,doac_fail},{},{}?.evkant + ?{diseases,doac_int},{},{}?.evkant),
%  ewarf = ({}.ewarf + e_warf),
%  e_warf = {get_warfarin}.empty,
%  edoac = ({}.edoac + e_doac),
%  e_doac = (?{},{dabigatran},{get_apixaban}?.e_doacfail + ?{},{apixaban},{get_dabigatran}?.e_doacfail),
%  e_doacfail = (?{doac_fail},{},{stop_doac}?.evkant + ?{},{doac_fail},{}?.e_doacfail),
%  evkant = ({}.evkant + e_vkant),
%  e_vkant = {get_vkant}.empty,	
%  ghyper = (?{},{hyper},{}?.ghyper + ?{hyper},{},{}?.g1),
%  g1 = (?{diabete},{},{}?.g2 + ?{below55},{diabete,origin},{}?.g2 + ?{},{below55,diabete},{}?.g3 + ?{origin},{diabete},{}?.g3),
%  g2 = ({}.g2 + <1,e_acei>.g4 + <1,e_arb>.g5),
%  g3 = ({}.g3 + <1,e_cbb>.g6),
%  g4 = ({}.g4 + <1,e_cbb>.g7 + <1,e_td>.g8),
%  g5 = ({}.g5 + <1,e_cbb>.g9 + <1,e_td>.g10),
%  g6 = ({}.g6 + <1,e_acei>.g7 + <1,e_arb>.g9 + <1,e_td>.g11),
%  g7 = ({}.g7 + <1,e_arb>.etd + <1,e_td>.earb),
%  g8 = ({}.g8 + <1,e_arb>.ecbb + <1,e_cbb>.earb),
%  g9 = ({}.g9 + <1,e_acei>.etd + <1,e_td>.eacei),
%  g10 = ({}.g10 + <1,e_acei>.ecbb + <1,e_cbb>.eacei),
%  g11 = ({}.g11 + <1,e_acei>.earb + <1,e_arb>.eacei),
%  ecbb = ({}.ecbb + e_cbb),
%  eacei = ({}.eacei + e_acei),
%  e_acei = (?{},{captopril},{get_benazepril}?.empty + ?{},{benazepril},{get_captopril}?.empty),
%  earb = ({}.earb + e_arb),
%  e_arb = (?{},{irbesartan},{get_olmesortan}?.empty + ?{},{olmesortan},{get_irbesartan}?.empty),
%  etd = ({}.etd + e_td),
%  e_td = (?{},{chlorothiazide},{get_indapamide}?.empty + ?{},{indapamide},{get_chlorothiazide}?.empty),
%  k_doac = (?{doac_test},{},{doac_ok}?.empty + ?{doac_test},{},{doac_fail}?.empty + ?{},{doac_test},{}?.k_doac),
%  empty = {}.empty,
%  kafib = ({afib}.empty + empty),
%  khf = ({has_fib}.empty + empty),
%  khr = ({heart_rate}.empty + empty),
%  kcons = ({consensus_acei}.empty + empty),
%  kage = ({over75}.empty + {below55}.empty + empty),
%  kdiabete = ({diabete}.empty + empty),
%  kdoacint = ({doac_int}.empty + empty),
%  khyper = ({hyper}.empty + empty),
%  korigin = ({origin}.empty + empty) ]").
%\end{verbatim}

\subsection{Auxiliary Material for the Protein Signaling Networks Case Study}\label{app:psn}

The \BioResolve specification for the protein signaling networks case study presented in~\cite{DBLP:conf/cmsb/BallisBFO24} is reported in \Cref{fig:bioresolve:psn}. The corresponding experimentation with \GROOVE has been discussed in \Cref{sec:ccReact}.


\begin{figure}[t]
\begin{minipage}{0.9\linewidth}
\footnotesize
\begin{verbatim}
myentities([]).

myreactions([
    react([akt],[],[akt]),
    react([erbb3],[],[akt]),
    react([mtor],[],[akt]),
    react([pdk1],[],[akt]),
    react([erbb1],[e,p],[erbb1]),
    react([egf],[e,p],[erbb1]),
    react([plcg],[e,p],[erbb1]),
    react([erbb2],[e,t,p],[erbb2]),
    react([egf],[e,t,p],[erbb2]),
    react([erbb3],[e,t,p],[erbb2]),
    react([erbb3],[e,p],[erbb3]),
    react([hrg],[e,p],[erbb3]),
    react([erk12],[],[erk12]),
    react([egf],[],[erk12]),
    react([p],[],[erk12]),
    react([mek12],[],[erk12]),
    react([mek12],[],[mek12]),
    react([erbb1],[],[mek12]),
    react([erbb2],[],[mek12]),
    react([erbb3],[],[mek12]),
    react([mtor],[],[mtor]),
    react([p],[],[mtor]),
    react([akt],[],[mtor]),
    react([p70s6k],[],[p70s6k]),
    react([akt],[],[p70s6k]),
    react([mtor],[],[p70s6k]),
    react([erk12],[],[p70s6k]),
    react([pdk1],[],[pdk1]),
    react([erbb1],[],[pdk1]),
    react([erbb2],[],[pdk1]),
    react([erbb3],[],[pdk1]),
    react([mek12],[],[pdk1]),
    react([pkca],[],[pkca]),
    react([plcg],[],[pkca]),
    react([plcg],[],[plcg]),
    react([egf],[],[plcg]),
    react([erbb1],[],[plcg]),
    react([erbb2],[],[plcg]),
    react([erbb3],[],[plcg]) ]).

myenvironment("[
    k = {egf,hrg}.k,
    ket = {e,t}.ket,
    korep = ({e}.korep + {p}.korep),
    korept = ({e}.korept + {p}.korept + {t}.korept),
    kge = (?{erbb1},{},{e}?.kge 
          + ?{erbb2},{},{e}?.kge 
          + ?{},{erbb1,erbb2},{}?.kge) ]").

\end{verbatim}
\end{minipage}
\caption{\BioResolve implementation of the protein signaling network case study from \Cref{sec:ccReact}.}
\label{fig:bioresolve:psn}
\end{figure}

\subsection{Auxiliary Material for the T Cell Differentiation Case Study}\label{app:maude}

The \BioResolve specification derived from the Boolean network model (available at \cite{ModelCellCollective}, see \Cref{fig:boolean-formulas}) of the T cell differentiation case study from \cite{puniya2018mechanistic}, and exploited in \cite{datamod2023} is reported in \Cref{fig:bioresolve:tcell}. The corresponding experimentation with \GROOVE has been discussed in \Cref{sec:datamod2023}.




\begin{figure}[t]
	\begin{center}
		\includegraphics[width=0.49\textwidth]{figs-datamod2023/TcellBN.png}
	\end{center}
%\fontsize{8}{8}
%\begin{verbatim}
% Jak1 = IFNgR and not SOCS1 
% IL21 = STAT3 and NFAT
%IL18R = IL18 and IL12 and not STAT6 
%SOCS1 = STAT1 or Tbet 
%  IL6 = RORgt 
%STAT5 = IL2R 
% IL17 = (RORgt and not STAT1) 
%        or (STAT3 and IL17 and IL23R and not STAT1 and not STAT5) 
%STAT4 = IL12R and IL12 and not GATA3 
%IFNgR = (IFNg_e and NFAT) or (IFNg and NFAT) 
%STAT6 = IL4R and not IFNg and not SOCS1 
%GATA3 = (STAT6 and NFAT and not TGFb and not RORgt and not Foxp3 
%        and not Tbet) or (GATA3 and not Tbet) 
%        or (STAT5 and not TGFb and not RORgt and not Foxp3 and not Tbet) 
%  IL4 = GATA3 and NFAT and not STAT1 
% NFkB = IRAK and not Foxp3 
%  IL2 = NFAT and NFkB and not Tbet 
%IL23R = (IL23 and STAT3 and not Tbet) or STAT3 
% Tbet = (STAT4 and not RORgt and not Foxp3) 
%        or (STAT1 and not RORgt and not Foxp3) 
%        or (Tbet and not IL12 and not IFNg and not RORgt and not Foxp3) 
%TGFbR = TGFb and NFAT 
%RORgt = TGFbR and ((STAT3 and IL21R) or (STAT3 and IL6R)) and not Tbet 
%        and not GATA3 and not Foxp3 
% IL6R = IL6 or IL6_e 
%IL21R = IL21 
%Foxp3 = (TGFbR and not (IL6R and STAT3) and not IL21R and not GATA3) 
%        or (STAT5 and not (IL6R and STAT3) and not IL21R and not GATA3) 
% IRAK = IL18R 
%IL12R = (IL12 and NFAT) or (STAT4 and not GATA3) or Tbet or (TCR and not GATA3) 
% IL2R = IL2 and NFAT 
%STAT3 = IL21R or IL23R or IL6R 
% IFNg = NFkB or (STAT4 and NFkB and NFAT and not STAT3 and not STAT6) 
%        or (Tbet and not STAT3) 
% NFAT = TCR and not Foxp3 
%STAT1 = (IL27 and NFAT) or Jak1 
% IL4R = (IL4 and not SOCS1) or IL4_e 
%\end{verbatim}
	\caption{Boolean updates of the T Cell differentiation model from \cite{puniya2018mechanistic}, available at \cite{ModelCellCollective}.}
	\label{fig:boolean-formulas}
\end{figure}

\begin{figure}[t]
\begin{minipage}{0.9\linewidth}
\footnotesize
\begin{verbatim}
myreactions([
    react([stat5],[gata3,il21r,il6r],[foxp3]),
    react([stat5],[gata3,il21r,stat3],[foxp3]),
    react([tgfbr],[gata3,il21r,il6r],[foxp3]),
    react([tgfbr],[gata3,il21r,stat3],[foxp3]),
    react([gata3],[tbet],[gata3]),
    react([nfat,stat6],[foxp3,rorgt,tbet,tgfb],[gata3]),
    react([stat5],[foxp3,rorgt,tbet,tgfb],[gata3]),
    react([nfat,nfkb,stat4],[stat3,stat6],[ifng]),
    react([nfkb],[],[ifng]),
    react([tbet],[stat3],[ifng]),
    react([ifng,nfat],[],[ifngr]),
    react([ifnge,nfat],[],[ifngr]),
    react([il12,nfat],[],[il12r]),
    react([tbet],[],[il12r]),
    react([tcr],[gata3],[il12r]),
    react([il17,il23r,stat3],[stat1,stat5],[il17]),
    react([rorgt],[stat1],[il17]),
    react([il12,il18],[stat6],[il18r]),
    react([nfat,nfkb],[tbet],[il2]),
    react([nfat,stat3],[],[il21]),
    react([il21],[],[il21r]),
    react([il23,stat3],[tbet],[il23r]),
    react([stat3],[],[il23r]),
    react([il2,nfat],[],[il2r]),
    react([stat4],[gata3],[il2r]),
    react([gata3,nfat],[stat1],[il4]),
    react([il4],[socs1],[il4r]),
    react([il4e],[],[il4r]),
    react([rorgt],[],[il6]),
    react([il6],[],[il6r]),
    react([il6e],[],[il6r]),
    react([il18r],[],[irak]),
    react([ifngr],[socs1],[jak1]),
    react([tcr],[foxp3],[nfat]),
    react([irak],[foxp3],[nfkb]),
    react([il21r,stat3,tgfbr],[foxp3,gata3,tbet],[rorgt]),
    react([il6r,stat3,tgfbr],[foxp3,gata3,tbet],[rorgt]),
    react([stat1],[],[socs1]),
    react([tbet],[],[socs1]),
    react([il27,nfat],[],[stat1]),
    react([jak1],[],[stat1]),
    react([il21r],[],[stat3]),
    react([il23r],[],[stat3]),
    react([il6r],[],[stat3]),
    react([il12,il12r],[gata3],[stat4]),
    react([il2r],[],[stat5]),
    react([il4r],[ifng,socs1],[stat6]),
    react([stat1],[foxp3,rorgt],[tbet]),
    react([stat4],[foxp3,rorgt],[tbet]),
    react([tbet],[foxp3,ifng,il12,rorgt],[tbet]),
    react([nfat,tgfb],[],[tgfbr]) ]).
    
\end{verbatim}
\end{minipage}
\caption{\BioResolve implementation of the T cell case study from \Cref{sec:datamod2023}.}
\label{fig:bioresolve:tcell}
\end{figure}

\subsection{Auxiliary Material for the \GROOVE experiments}\label{app:groove}

To replicate the \GROOVE experiments reported in \Cref{sec:RS2GTS} (for the toy running example) and \Cref{sec:experiments}, we have included the following supplementary resources with this paper:
\begin{itemize}
\item The rule systems described in \Cref{sec:RS2GTS};
\item The start graphs derived from the \BioResolve specifications in this appendix (\ref{app:running}--\ref{app:maude});
\item Instructions for calling the \GROOVE generator so as to reproduce all the exploration runs, occurrence graphs and model checking results (using \href{https://github.com/nl-utwente-groove/code/releases/tag/release-7_4_3}{\GROOVE version 7.4.3}).
\end{itemize}



\end{appendices}


\bibliography{references}



\end{document}
